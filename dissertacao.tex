%% Exemplo de utilizacao do estilo de formatacao normas-utf-tex (http://normas-utf-tex.sourceforge.net)
%% Autores: Hugo Vieira Neto (hvieir@utfpr.edu.br)
%%          Diogo Rosa Kuiaski (diogo.kuiaski@gmail.com)
%% Colaboradores:
%%          César M. Vargas Benitez <cesarvargasb@gmail.com>
%%          Marcos Talau <talau@users.sourceforge.net>

\documentclass[openright]{normas-utf-tex} %openright = o capitulo comeca sempre em paginas impares
%\documentclass[oneside]{normas-utf-tex} %oneside = para dissertacoes com numero de paginas menor que 100 (apenas frente da folha) 

\usepackage[alf,abnt-emphasize=bf,bibjustif,recuo=0cm, abnt-etal-cite=2, abnt-etal-list=99]{abntcite} %configuracao correta das referencias bibliograficas.

\usepackage[brazil]{babel} % pacote portugues brasileiro
%\usepackage[latin1]{inputenc} % pacote para acentuacao direta ISO-8859-1
\usepackage[utf8]{inputenc}
\usepackage{amsmath,amsfonts,amssymb} % pacote matematico
\usepackage{graphicx} % pacote grafico
\usepackage{times} % fonte times

%Podem utilizar GEOMETRY{...} para realizar pequenos ajustes das margens. Onde, left=esquerda, right=direita, top=superior, bottom=inferior. P.ex.:
%\geometry{left=3.0cm,right=1.5cm,top=4cm,bottom=1cm} 

% ---------- Preambulo ----------
\instituicao{Universidade Tecnológica Federal do Paraná} % nome da instituicao
\programa{Programa de Pós-graduação em Engenharia Elétrica e Informática Industrial} % nome do programa
\area{Informática Industrial} % [Engenharia Biom\'edica] ou [Inform\'atica Industrial] ou [Telem\'atica]

\documento{Dissertação} % [Disserta\c{c}\~ao] ou [Tese]
\nivel{Mestrado} % [Mestrado] ou [Doutorado]
\titulacao{Mestre} % [Mestre] ou [Doutor]


\titulo{\MakeUppercase{Título em Português}} % titulo do trabalho em portugues
\title{\MakeUppercase{Title in english}} % titulo do trabalho em ingles

\autor{autor do trabalho} % autor do trabalho
\cita{TRABALHO, autor} % sobrenome (maiusculas), nome do autor do trabalho

\palavraschave{palavra1, palavra2, palavra3}
\keywords{word1, word2, word3} 

\comentario{\UTFPRdocumentodata\ apresentada ao \UTFPRprogramadata\ da \ABNTinstituicaodata\ como requisito parcial para obtenção do grau de ``\UTFPRtitulacaodata\ em Ciências'' -- \'Area de Concentração: \UTFPRareadata.}

\orientador{Nome do Orientador} % nome do orientador do trabalho
\coorientador[Co-orientadores:]{Nome do Coorientador} % <- no caso de co-orientadora, usar esta sintaxe
%\coorientadorb{}	% este comando inclui o nome do 2o co-orientador

\local{Curitiba} % cidade
\data{\the\year} % ano automatico


\usepackage{algorithm}
\usepackage{algpseudocode} % já carrega automaticamente \usepackage{algorithmicx
\floatname{algorithm}{Algoritmo}

%%%%%%%%%%%%%%%%%%%%%%%%%%%%%%%%%%%%%%%%%%%%%%%%%%%%%%%%%%%%%%
%
%		Inicio do Documento
%
%%%%%%%%%%%%%%%%%%%%%%%%%%%%%%%%%%%%%%%%%%%%%%%%%%%%%%%%%%%%%%
\begin{document}
\capa % geracao automatica da capa
\folhaderosto % geracao automatica da folha de rosto
%\termodeaprovacao % <- ainda a ser implementado corretamente

% dedicatria (opcional)
\begin{dedicatoria}
Texto da dedicatória.
\end{dedicatoria}

% agradecimentos (opcional)
\begin{agradecimentos}
Texto dos agradecimentos.
\end{agradecimentos}

% epigrafe (opcional)
\begin{epigrafe}
Texto da epígrafe.
\end{epigrafe}

%resumo
\begin{resumo}
Texto do resumo (máximo de 500 palavras).
\end{resumo}

\begin{abstract}
Abstract text (maximum of 500 words).
\end{abstract}

% listas (opcionais, mas recomenda-se a partir de 5 elementos)
\listadefiguras % geracao automatica da lista de figuras
\listadetabelas % geracao automatica da lista de tabelas
\listadesiglas % geracao automatica da lista de siglas
\listadesimbolos % geracao automatica da lista de simbolos
\sumario 

%%%%%%%%%%%%%%%%%%%%%%%%%%%%%%%%%%%%%%%%%%%%%%%%%%%%%%%%%%%%%%
%
%		INTRODUÇÃO
%
%%%%%%%%%%%%%%%%%%%%%%%%%%%%%%%%%%%%%%%%%%%%%%%%%%%%%%%%%%%%%%
\chapter{Introdução}



\section{Objetivos}



\section{Contribuições}



\section{Organização do Trabalho}
Revisão bibliográfica esta descrita no Capítulo \ref{chap:revisao},...\ref{chap:desenv}, \ref{chap:resultados}, \ref{chap:conclusao}.

%%%%%%%%%%%%%%%%%%%%%%%%%%%%%%%%%%%%%%%%%%%%%%%%%%%%%%%%%%%%%%
%
%		REVISÃO BIBLIOGRÁFICA
%
%%%%%%%%%%%%%%%%%%%%%%%%%%%%%%%%%%%%%%%%%%%%%%%%%%%%%%%%%%%%%%
\chapter{Revisão Bibliográfica}\label{chap:revisao}



%%%%%%%%%%%%%%%%%%%%%%%%%%%%%%%%%%%%%%%%%%%%%%%%%%%%%%%%%%%%%%
%
%		DESENVOLVIMENTO
%
%%%%%%%%%%%%%%%%%%%%%%%%%%%%%%%%%%%%%%%%%%%%%%%%%%%%%%%%%%%%%%
\chapter{Desenvolvimento}\label{chap:desenv}


%%%%%%%%%%%%%%%%%%%%%%%%%%%%%%%%%%%%%%%%%%%%%%%%%%%%%%%%%%%%%%
%
%		EXPERIMENTOS E RESULTADOS
%
%%%%%%%%%%%%%%%%%%%%%%%%%%%%%%%%%%%%%%%%%%%%%%%%%%%%%%%%%%%%%%
\chapter{Experimentos e Resultados} \label{chap:resultados}



%%%%%%%%%%%%%%%%%%%%%%%%%%%%%%%%%%%%%%%%%%%%%%%%%%%%%%%%%%%%%%
%
%		CONCLUSÂO
%
%%%%%%%%%%%%%%%%%%%%%%%%%%%%%%%%%%%%%%%%%%%%%%%%%%%%%%%%%%%%%%
\chapter{Conclusão}\label{chap:conclusao}

%%%%Contexto do trabalho


%%%%Objetivos


%%%%Explique os experimento


%%%%Conclusões


%%%%Indicativo de trabalho futuros





%%%%%%%%%%%%%%%%%%%%%%%%%%%%%%%%%%%%%%%%%%%%%%%%%%%%%%%%%%%%%%
%---------- Referencias ----------
%\ABNTbibliographyname
\bibliography{referencias}

%---------- Apendices (opcionais) ---------- Utilizar para adicionar informações da pesquisa/resultados
\apendice
\chapter{Comparação com a Literatura}

%Use o comando {\ttfamily \textbackslash apendice} e depois comandos {\ttfamily \textbackslash chapter\{\}} para gerar títulos de apêndices.


% ---------- Anexos (opcionais) ---------- Utilizar para Explicar assuntos diversos
\anexo
\chapter{Nome do Anexo}

%Use o comando {\ttfamily \textbackslash anexo} e depois comandos {\ttfamily \textbackslash chapter\{\}} para gerar títulos de anexos.


%-------- Citacoes ---------
% - Utilize o comando \citeonline{...} para citacoes com o seguinte formato: Autor et al. (2011).
% Este tipo de formato eh utilizado no comeco do paragrafo. P.ex.: \citeonline{autor2011}
% - Utilize o comando \cite{...} para citacoeses no meio ou final do paragrafo. P.ex.: \cite{autor2011}

%\newpage
\clearpage

%\begin{document}

\sigla{AEMO}{Algoritmo Evolucionário Multi-Objetivo.}
\sigla{AG}{Algoritmo Genético.}
\sigla{DRA}{\emph{Dynamical Resource Allocation}.}
\sigla{ED}{Evolução Diferencial.}
\sigla{FIR}{\emph{Fitness Improvement Rate}.}
\sigla{FRR}{\emph{Fitness Rate Rank}.}
\sigla{IBEA}{\emph{Indicator based Evolutionary Algorithm}.}
\sigla{IGD}{\emph{Inverted Generational Distance}.}
\sigla{IRace}{\emph{Iterated Race}.}
\sigla{HH}{Hiper Heurística.}
\sigla{HV}{Hiper Volume.}
\sigla{MAB}{\emph{Multi Armed Bandit}.}
%\sigla{MOEA}{Algoritmo Evolucionário Multi-Objetivo.}%mesmo AEMO mas em inglês
\sigla{MOEAD}{\emph{Multi-Objective Evolutionary Algorithm base on Decomposition}.}
\sigla{NSGAII}{\emph{Nondominated Sorting Genetic Algorithm version II}.}
\sigla{POM}{Problema de Otimização Multi-objetivo.}
\sigla{SaDE}{\emph{Self-adaptive differential evolution}.}
\sigla{SPEA2}{\emph{Strength Pareto Evolutionary Algorithm 2}.}
\sigla{UCB}{\emph{Upper Confidence Bound}.} 

%\end{document}

% POM
\simbolo{$x_i^{(L)}$}{Restrição inferior para a variável de decisão i.}
\simbolo{$x_i^{(U)}$}{Restrição superior para a variável de decisão i.}
\simbolo{$\Omega$}{Conjunto finito de soluções factíveis.}

%  ED
\simbolo{$F$}{Fator de Mutação.}
\simbolo{$CR$}{Valor de Crossover.}
\simbolo{$NP$}{Tamanho da População.}

%	MOEA/D
\simbolo{$N$}{Número de Subproblemas.}
\simbolo{$z^*$}{Ponto Ideal Empírico.}
\simbolo{$\delta$}{Valor que determina seleção de escopo.}
\simbolo{$\lambda$}{Vetor peso.}
\simbolo{$nr$}{Número máximo de inserções de um indivíduo no escopo.}

%	DRA
\simbolo{$\pi^i$}{Valor de utilidade do subproblema $i$.}
\simbolo{$\Delta^i$}{É a diminuição relativa do valor da função objetivo do subproblema $i$.}


\simbolo{$g^{tche1}$}{Função de agregação Tchebycheff1 (Equação \ref{eq:tche}).}
\simbolo{$g^{tche2}$}{Função de agregação Tchebycheff2 (Equação \ref{eq:tche2}).}
\simbolo{$g^{pbi}$}{Função de agregação PBI (Equação \ref{eq:PBI}).}


%UCB
\simbolo{$C$}{Valor que controla o \emph{trade-off} entre exploração e intensificação.}
\simbolo{$D$}{Valor de Decaimento.}
\simbolo{$op$}{Operador selecionado pelo UCB.}
\simbolo{$n_op$}{Número de vezes que o operador $op$ foi aplicado na Janela de tempo.}
\simbolo{$SW$}{\emph{Sliding Window}, onde os dados das execuções serão armazenados.}

%\simbolo{$$}{}


\end{document}



